\documentclass[10pt]{article}

\usepackage{lmodern}
\usepackage[T1]{fontenc}
\usepackage{amsmath,amsthm,amssymb}
\usepackage{mathpartir}
\usepackage{stmaryrd}
\usepackage{xcolor}
\usepackage[shortlabels]{enumitem}
\usepackage{listings}

% copied from package braket
% edited to remove whitespace inside braces
% and to simplify the definition of \Set because we don't use double-vert
{\catcode`\|=\active
  \xdef\set{\protect\expandafter\noexpand\csname set \endcsname}
  \expandafter\gdef\csname set \endcsname#1{\mathinner
    {\lbrace{\mathcode`\|32768\let|\midvert #1}\rbrace}}
  \xdef\Set{\protect\expandafter\noexpand\csname Set \endcsname}
  \expandafter\gdef\csname Set \endcsname#1{\left\{%
     {\mathcode`\|32768\let|\SetVert #1}\right\}}
}
\def\midvert{\egroup\mid\bgroup}
\makeatletter
\def\mid@vertical{\mskip1mu\vrule\mskip1mu}
\def\SetVert{{\egroup\;\mid@vertical\;\bgroup}}
\makeatother

\newtheorem{theorem}{Theorem}

\usepackage[margin=1.5in]{geometry}
\usepackage{changepage}

%\usepackage{fancyhdr}
%\pagestyle{fancy}
%\fancyhead[C]{}
%\fancyhead[R]{}
%\setlength{\headheight}{14pt}

\usepackage{lipsum}

\newcommand{\mypyvy}{\texttt{mypyvy}}
\title{The \mypyvy{} Language for\\First-order Symbolic Transition Systems}
\author{James R. Wilcox \and Yotam M. Y. Feldman \and Oded Padon \and Sharon Shoham \and And Friends Who Agree to Join}
\date{}

\begin{document}
\maketitle

\section*{Introduction}

\begin{verbatim}
- mathematical transition systems
- first-order symbolic transition systems "on paper"
\end{verbatim}

\section*{Expressing Transition Systems in \mypyvy}

\begin{verbatim}
- sort
- immutable/mutable relation/constant/function
- init
- transition
- zero/one/twostate definitions
- attributes
- typechecking/inference
- implicit quantification of capitalized vars at outer scope
- note on implicit existential on transition params, but *not* defn params
- note on modifies clauses/frame conjuncts
\end{verbatim}

\section*{Queries on Transition Systems}

\begin{verbatim}
- trace/bmc; the trace declaration; sat/unsat qualifier
- how to read states printed by mypyvy
- verify: invariant/safety
- zero/one/twostate theorem
- side note: custom printers using attributes
- updr
\end{verbatim}

\section*{Internals of \mypyvy}

\begin{verbatim}
- a tour of main()
- the mental model of k-state formulas (correctly handling immutable)
  - evaluating a k-state formula on a trace
- philosophy on interacting with z3, the Solver class
- how to write a mypyvy "plugin"
- syntax.the_program and its consequences
\end{verbatim}

\section*{Past, Present, and Future Work}

\begin{verbatim}
- our port of the raft proof
- yotam's cav19
- jason's plid20
- pd
- derived relations?
- yotam's looking back algorithm or whatever it's called
- future directions in internals:
  - collapsing the many kinds of queries into one
  - handling transition declarations more uniformly (exists, modifies)
  - introducing a "logic" layer or other IR, or do less at z3 level
  - revisiting the "one program" mindset
\end{verbatim}

\section*{Conclusion}

\begin{verbatim}
- call to arms for collaborators, builders-on-toppers, and users
- vision blah blah about verification UX and "exploration" of a TS,
  saving progress from run to run, "workbench"
\end{verbatim}

\end{document}
